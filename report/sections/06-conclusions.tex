%%!TEX root=../main.tex

\section{Conclusion}
\subsection{Performance}
In term of CNN model, we can find that the performance of sparse data doesn't perform better than original data, 
especially for the province category. The reason behind it may because our original image is small enough 
and obtaining the sparse representation of a very small and binaries image cause an adverse effect on the 
info being retained in the data. In another words, causing a information loss in the sparse representation.
In the end, causing an accuracy loss in the classifier. 

Comparing SVM and CNN, the validation accuracies are similar. Therefore, applying sparse representation data
on SVM can also be an viable option. On top of that, SVM is more lightweight than CNN which could be used
easily on a resource hungry environment, such as parking-lot cameras. Also, it takes less time to train the data. 

If the scenario requires a quick response time, such as the camera on highway, CNN will a better choice. 

\subsection{System Proposal}
In this paper, we have successful proposed a workflow of an end-to-end license plate
recognition system. Starting with image pre-processing, sparse dictionary learning and sparse coding problem to 
classifier model choice. We have also demonstrated the accuracy of our workflow can be up to 99.05\% accuracy 
with the CCPD2019\cite{xu2018towards, li2017endtoend} dataset containing over thousands of images.
Even if the sparse representation is not hitting as high accuracy as expected, the accuracy achievable is around
97.46\% which is still a significant number. Also the sparse nature of this representation allow rooms for 
further compression for data storage. 




