%%!TEX root=../main.tex

License plate recognition system is one of the key pillars towards supporting 
intelligent traffic management. Since plate recognition system allow employing
various image processing techniques and a complicated process, the research interest
of this topic is huge.

The recognition process often include three stages. Firstly, the image of a vehicle 
is being de-noise and turn into grey-scale. Secondly, the polygon region of the 
plate is detected and character separation is being done. Thirdly, the characters
are being identified using intelligent classification methods. There are high 
demands of the license plate recognition systems, both public and private. 

In this paper, an system is proposed that can recognize plates in China. 
License plate images in the dataset are taken at different angles, distances, 
vehicle conditions and periods of the day. This allows the system to incorporate
various illumination conditions into account. Since license plate recognition 
usually takes image or video source as input, using sparse representation can 
be a promising direction. 

The plate is localized using various contour detection methods and the plate features. 
Plate structure and government regulation are used for character segmentation. 
Finally, character recognition is done by convolution dictionary learning and 
convolution sparse coding. A comparison between normal convolution neural network 
and sparse representation is also included in this paper.