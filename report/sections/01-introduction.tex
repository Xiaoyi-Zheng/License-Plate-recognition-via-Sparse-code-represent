%%!TEX root=../main.tex

\section{Introduction}
% TODO: add lpr NN reference
License plate recognition is a common component in traffic management system
and is crucial to a successful intelligent city infrastructure. The key idea is 
that automatic license plate recognition can significantly reduce the need of 
human labour and thus reducing the cost of urban city maintenance. Since this
fall into the huge category of image recognition, there is a huge interest 
in leveraging the ability of neural network to improve the accuracy and to 
replace the heuristic method using deep learning.

Sparse representation has been used in a large variety of research, including
de-noising\cite{4011956}, pattern recognition\cite{CGV-058}, image up-scaling\cite{SparesuperResolution}
and facial recognition\cite{SparseFace}. A key idea in sparse coding is how to select
or "learn" a dictionary. In this paper, the CCPD2019\cite{xu2018towards} 
dataset is being used as the main training data. Some of the early works has purposed
creating a representation using fixed-base\cite{5452966} while more recent research
has providing proofs that a learned dictionary can achieve better performance\cite{4011956}.
The original goal of the sparse coding problem is not targeted towards classification problem but
to signal representation and compression. However, we can perform classification 
by exploiting the discriminative nature of sparse representation. In this work, instead of selecting
a dictionary, we learned the dictionary using the training data themselves. With sufficient samples
from each class, it is possible to form a linear representation to present each samples. Yet,
the representation has to be sparse enough. The generated sparse representation is then directly
used to classify the samples. 

% TODO: add chinese license plate 
Sparse representation is well suited to reduce the complexity of the input dataset of 
images of license plate due to the inherit regular format of license 
plate following the local government law. The challenge presented here is that 
the license plate in China include Chinese characters presenting different provinces.
Even the characters come from a limited set, the chinese characters have more strokes
than common alphabet or numerical digits used in the western countries. Since most of
Chinese character recognition project are target towards hand-written images, they are
well suit for the purpose of this research.

A key limitation of using deep learning method for a license plate recognition system
is that the size of the incoming stream of data is huge compared to normal image recognition
or pattern detection. It is common to have thousand of traffic flows in a urban city 
which license plate recognition are needed in a lot of the corners of the city's highway
and road. Sending all the data to a data center for plate recognition is not an ideal
solution since a lot of the old highways don't have ideal telecommunication system built
in the infrastructure. Therefore a certain amount of light pre-processing is required 
on the embedded device in the monitoring camera. Using sparse representation can 
allow less amount of processing power of the distributed system and have part of the 
recognition implanted in camera system which has limited battery life and processing power.

The key contributions of this paper include:
\begin{itemize}
	\item Provide detail description of the end-to-end process from vehicle image
	to license plate digit output
	\item Provide a usage example of incorporating sparse encoding into license
	plate recognition
	\item Developed a license plate recognition targeting China license plate 
	format 
	\item Comparison of convolution dictionary learning (CDL) methods between 
	Convolution Basis Pursuit De-Noising (CBPDN)\cite{CDLreview2018} and Online 
	CDL with spatial masking\cite{onlineCDL}
	\item Comparison of convolution sparse coding (CDC) methods between 
	Single channel CSC\cite{CDLreview2018}, FISTA CBPDN solver\cite{SOREL201644} and Single channel CSC with 
	lateral inhibition\cite{CDLreview2018}
	\item Quantification of the accuracy-complexity trade-offs, runtime
	for neural network using data from the CCPD2019~\cite{xu2018towards} dataset
\end{itemize}

